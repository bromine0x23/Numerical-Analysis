\subsection{P269-4}

\renewcommand{\EPSILON}{{10}^{-5}}

用Romberg公式求积分,使误差不超过$\EPSILON$

\begin{enumerate}
%%%%%%%%%%%%%%%%%%%%%%%%%%%%%%%%
\renewcommand{\FX}{e^{-x^{2}}}
\renewcommand{\LRANGE}{0}
\renewcommand{\RRANGE}{1}

\item \begin{displaymath}
\frac{2}{\sqrt{\pi}} \INTRANGE{\FX}{x}
\end{displaymath}

\begin{SOLVE}
在MATLAB命令窗口执行
\begin{lstlisting}
>> f = @(x) 2 / sqrt(pi) * exp(- x .^ 2); a = 0; b = 1; epsilon = 1E-5;
>> I = m_romberg(f, a, b, epsilon)

I =

    0.8427

\end{lstlisting}
\end{SOLVE}
%%%%%%%%%%%%%%%%%%%%%%%%%%%%%%%%
\renewcommand{\FX}{x \sqrt{1 + x ^ 2}}
\renewcommand{\LRANGE}{0}
\renewcommand{\RRANGE}{3}

\item \begin{displaymath}
\INTRANGE{\FX}{x}
\end{displaymath}

\begin{SOLVE}
在MATLAB命令窗口执行
\begin{lstlisting}
>> f = @(x) 2 / sqrt(pi) * exp(- x .^ 2); a = 0; b = 1; epsilon = 1E-5;
>> I = m_romberg(f, a, b, epsilon)

I =

   10.2076

\end{lstlisting}
\end{SOLVE}
%%%%%%%%%%%%%%%%%%%%%%%%%%%%%%%%
\end{enumerate}
