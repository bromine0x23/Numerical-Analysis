\subsection{P270-5}

\renewcommand{\FX}{\frac{1}{x}}
\renewcommand{\LRANGE}{0}
\renewcommand{\RRANGE}{1}
\renewcommand{\EPSILON}{{10}^{-5}}

求积分
\begin{displaymath}
\INTRANGE{\FX}{x}
\end{displaymath}

\begin{enumerate}
%%%%%%%%%%%%%%%%%%%%%%%%%%%%%%%%
\item
\renewcommand{\N}{16}
使用复合梯形公式,$n = \N$
\begin{SOLVE}
在MATLAB命令窗口执行
\begin{lstlisting}
>> f = @(x) 1 ./ x; a = 0; b = 1; n = 16;
>> I = m_trapz(f, a + 1E-5, b, n)

I =

   3.1283e+03

\end{lstlisting}
\end{SOLVE}
%%%%%%%%%%%%%%%%%%%%%%%%%%%%%%%%
\item
\renewcommand{\N}{4}
使用复合Simpson公式,$n = \N$
\begin{SOLVE}
在MATLAB命令窗口执行
\begin{lstlisting}
>> f = @(x) 2 / sqrt(pi) * exp(- x .^ 2); a = 0; b = 1; epsilon = 1E-5;
>> I = m_simpson(f, a + 1E-5, b, n / 2)

I =

   8.3354e+03

\end{lstlisting}
\end{SOLVE}
%%%%%%%%%%%%%%%%%%%%%%%%%%%%%%%%
\item
Romberg算法
\begin{SOLVE}
在MATLAB命令窗口执行
\begin{lstlisting}
>> f = @(x) 1 ./ x; a = 0; b = 1; epsilon = 1E-5;
>> I = m_romberg(f, a + 1E-5, b, epsilon)

I =

   96.0977

\end{lstlisting}
\end{SOLVE}
%%%%%%%%%%%%%%%%%%%%%%%%%%%%%%%%
\item
将区间4等分,在每个区间上用两点Gauss公式计算,然后累加得积分值
\begin{SOLVE}
在MATLAB命令窗口执行
\begin{lstlisting}
>> f = @(x) 1 ./ x; a = 0; b = 1; n = 4;
>> I =  m_gauss2(f, a, b, n) 

I =

    4.3854

\end{lstlisting}
\end{SOLVE}

\end{enumerate}
