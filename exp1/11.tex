\subsection{P233-11}
\renewcommand{\FX}{\sin \frac{\pi}{2} x}
\renewcommand{\LRANGE}{-1}
\renewcommand{\RRANGE}{1}
\newcommand{\PA}{P_0}
\newcommand{\PB}{P_1}
\newcommand{\PC}{P_2}
\newcommand{\PD}{P_3}
\renewcommand{\PART}[1]{\frac{(f, P_{#1})}{(P_{#1}, P_{#1})} P_{#1}}
\renewcommand{\COEFFICIENTS}[8]{
\begin{align*}
\FUNCDOT{P_0}{P_0} & = \INTRANGE{\PA \cdot \PA}{x} = #1 &
\FUNCDOT{P_1}{P_1} & = \INTRANGE{\PB \cdot \PB}{x} = #2 \\
\FUNCDOT{P_2}{P_2} & = \INTRANGE{\PC \cdot \PC}{x} = #3 &
\FUNCDOT{P_3}{P_3} & = \INTRANGE{\PD \cdot \PD}{x} = #4 \\
\FUNCDOT{f}{P_0}   & = \INTRANGE{\FX \cdot \PA}{x} = #5 &
\FUNCDOT{f}{P_1}   & = \INTRANGE{\FX \cdot \PB}{x} = #6 \\
\FUNCDOT{f}{P_2}   & = \INTRANGE{\FX \cdot \PC}{x} = #7 &
\FUNCDOT{f}{P_3}   & = \INTRANGE{\FX \cdot \PD}{x} = #8
\end{align*}
}

$f(x) = \FX$在$\RANGE$上按Legendre多项式展开,求三次最佳平方逼近多项式
\begin{SOLVE}
\renewcommand{\PA}{1}
\renewcommand{\PB}{x}
\renewcommand{\PC}{\frac{3 x^2 - 1}{2}}
\renewcommand{\PD}{\frac{5 x^3 - 3 x}{2}}
令$\Phi(x)=\operatorname{span}\{P_0, P_1, P_2, P_3\}$,$\rho(x) \equiv 1$,其中
\begin{align*}
P_0 & = \PA &
P_1 & = \PB \\
P_2 & = \PC &
P_3 & = \PD
\end{align*}
计算可得
\COEFFICIENTS{2}{\frac{2}{3}}{\frac{2}{5}}{\frac{2}{7}}{0}{\frac{8}{\pi^2}}{0}{\frac{48(\pi^2 - 10)}{\pi^4}}
于是$f(x)$的三次最佳平方逼近多项式为\begin{align*}
P_3(x) & = \PART{0} + \PART{1} + \PART{2} + \PART{3} \\
       & = (\frac{420}{\pi^2} - \frac{4200}{\pi^4}) x^3 - (\frac{240}{\pi^2} - \frac{2520}{\pi^4}) x \\
       & = -0.5622 x^3 + 1.5532 x
\end{align*}
\end{SOLVE}