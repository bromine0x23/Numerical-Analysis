\subsection{P233-9}
求函数$f(x)$在指定区间上关于$\Phi(x)=\operatorname{span}\{1,x\}$的最佳平方逼近多项式
\renewcommand{\COEFFICIENTS}[5]{
\begin{align*}
b_{0}   = \FUNCDOT{1}{f} & = \INTRANGE{1 \cdot \FX}{x} = #1 &
b_{1}   = \FUNCDOT{x}{f} & = \INTRANGE{x \cdot \FX}{x} = #2 \\
g_{0,0} = \FUNCDOT{1}{1} & = \INTRANGE{1 \cdot 1}{x}   = #3 &
g_{0,1} = \FUNCDOT{1}{x} & = \INTRANGE{1 \cdot x}{x}   = #4 \\
g_{1,0} = \FUNCDOT{x}{1} & = \INTRANGE{x \cdot 1}{x}   = #4 &
g_{1,1} = \FUNCDOT{x}{x} & = \INTRANGE{x \cdot x}{x}   = #5 \\
\end{align*}
}

\newcommand{\EQUATION}{
\begin{displaymath}
\begin{pmatrix}
g_{0,0} &
g_{0,1} \\
g_{1,0} &
g_{1,1} \\
\end{pmatrix}
\begin{pmatrix}
a_{0} \\
a_{1} \\
\end{pmatrix}
=
\begin{pmatrix}
b_{0} \\
b_{1} \\
\end{pmatrix}
\end{displaymath}
}

\newcommand{\SOLUTION}[4]{
\begin{align*}
a_{0}  & = #1 &
a_{1}  & = #2 \\
\end{align*}
\begin{displaymath}
P_1(x) = (#2) x + (#1) = #4 x #3 \\
\end{displaymath}
}

\newcommand{\PATTERN}[9]{
$f(x) = \FX, x \in [\LRANGE, \RRANGE]$
\begin{SOLVE}
设$f(x)$的最佳平方逼近多项式为$P_1(x)$,取$\rho(x) \equiv 1$,有
\COEFFICIENTS{#1}{#2}{#3}{#4}{#5}
由
\EQUATION
解得
\SOLUTION{#6}{#7}{#8}{#9}
\end{SOLVE}
}

%\renewcommand{\FX}{\frac{1}{x}}
%\renewcommand{\LRANGE}{1}
%\renewcommand{\RRANGE}{3}
%(1) \PATTERN{\ln 2}{2}{2}{4}{\frac{26}{3}}{\frac{13 \ln 3}{2} - 6}{3 - 3 \ln 3}

%\renewcommand{\FX}{e^x}
%\renewcommand{\LRANGE}{0}
%\renewcommand{\RRANGE}{1}
%(2) \PATTERN{e - 1}{1}{1}{\frac{1}{2}}{\frac{1}{3}}{4 e - 10}{18 - 6 e}

\renewcommand{\FX}{\cos {\pi x}}
\renewcommand{\LRANGE}{0}
\renewcommand{\RRANGE}{1}
(3) \PATTERN{0}{-\frac{2}{\pi^2}}{1}{\frac{1}{2}}{\frac{1}{3}}{\frac{12}{\pi^2}}{-\frac{24}{\pi^2}}{+ 1.2159}{- 2.4317}

\renewcommand{\FX}{\ln x}
\renewcommand{\LRANGE}{1}
\renewcommand{\RRANGE}{2}
(4) \PATTERN{\ln 4 - 1}{\ln 4 - \frac{3}{4}}{1}{\frac{3}{2}}{\frac{7}{3}}{10 \ln 4 - \frac{29}{2}}{9 - 6 \ln 4}{-0.6371}{0.6822}